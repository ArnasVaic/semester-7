\documentclass[11pt]{article}

\usepackage{graphicx}
\usepackage{amsmath}
\usepackage{mathtools}
\usepackage{amssymb}
\usepackage[T1]{fontenc}
\usepackage[lithuanian]{babel}
\usepackage{listings}
\usepackage{subfigure}

% Margins
\topmargin=-0.45in
\evensidemargin=0in
\oddsidemargin=0in
\textwidth=6.5in
\textheight=9.0in
\headsep=0.25in
    
\title{ Pirmas laboratorinis darbas}
\author{ Arnas Vaicekauskas }
\date{\today}

\begin{document}
\maketitle

\section{Pirmas pratimas}

Turime paprastąją pirmos eilės diferencialinę lygtį

$$
2+y^2+\sqrt{5-x^2}yy'=0
$$


\subsection{Sprendimas}

% \setlength{\jot}{10px}
Kintamieji atsiskiria atlikus porą elementarių pertvarkymų:

\begin{equation}
\begin{split}
\sqrt{5-x^2}y\frac{dy}{dx}&=-(y^2+2) \\
\frac{ydy}{-(y^2+2)}&=\frac{dx}{\sqrt{5-x^2}}
\end{split}
\end{equation}

Integruojame:
\begin{equation}
    \begin{split}
        \int\frac{ydy}{-(y^2+2)}&=\int\frac{dx}{\sqrt{5-x^2}} \\
        -\frac{1}{2}\int\frac{d(y^2+2)}{y^2+2}&=\arcsin{\frac{x}{\sqrt{5}}}+C\\
        -\frac{1}{2}\ln|y^2+2|&=\arcsin{\frac{x}{\sqrt{5}}}+C\\
        \ln|y^2+2|&=C-2\arcsin{\frac{x}{\sqrt{5}}}\\
        y^2&=Ce^{-2\arcsin{\frac{x}{\sqrt{5}}}} - 2
    \end{split}
\end{equation}

% , C\in\mathbb R_{\ge 0}

Reikėtų atkreipti dėmesį, kad $y^2\geqslant 0$ todėl 
nevisos $C$ reikšmės bus tinkamos. Galime rasti mažiausią $C$
reikšmę su kuria lygtis turės realių sprendinių:

\begin{equation}
\max_{x\in[-\sqrt{5}, \sqrt{5}]} e^{-2\arcsin(\frac{x}{\sqrt 5})} = e^{\pi}
\end{equation}

tai

\begin{equation}
    \begin{split}
        Ce^\pi\geqslant2\\
        C\geqslant2e^{-\pi}
    \end{split}
\end{equation}


\newpage
\subsection{Tikrinimas}

Prieš tikrindami randame išvestinę "patogesnėje" išraiškoje:

\begin{equation}
    \begin{split}
        (y^2)'&=(Ce^{-2\arcsin{\frac{x}{\sqrt{5}}}} - 2)' \\
        2yy'&=\frac{-2C}{\sqrt{5-x^2}}e^{-2\arcsin{\frac{x}{\sqrt{5}}}}\\
        yy'&=\frac{-C}{\sqrt{5-x^2}}e^{-2\arcsin{\frac{x}{\sqrt{5}}}}
    \end{split}
\end{equation}

tada

\begin{equation}
    \begin{split}
        2+y^2+\sqrt{5-x^2}yy'=0\\
        2+Ce^{-2\arcsin{\frac{x}{\sqrt{5}}}} - 2+\sqrt{5-x^2}\frac{-C}{\sqrt{5-x^2}}e^{-2\arcsin{\frac{x}{\sqrt{5}}}}=0\\
        2-2+Ce^{-2\arcsin{\frac{x}{\sqrt{5}}}}-Ce^{-2\arcsin{\frac{x}{\sqrt{5}}}}=0\\
        0\equiv0
    \end{split}
\end{equation}

\subsection{Palyginimas}

Sprendiniai palyginimui buvo rasti naudojant 
python programavimo kalbą su sympy paketu. 
Toliau pavaizduotas kodas naudotas rasti sprendinį:

\begin{lstlisting}[language=Python]
from sympy import *
x = symbols('x')
y = Function('y')(x)
equation = Eq(2 + y**2 + y.diff(x) * y * sqrt(5 - x**2), 0)
solutions = dsolve(equation)
display(solutions[0], solutions[1])
\end{lstlisting}

Rastas sprendinys turi forma:

\begin{equation}
y=\pm\sqrt{Ce^{-2\arcsin(\frac{\sqrt{5}x}{5})}-2}
\end{equation}

Akivaizdu, kad šis sprendinys ekvivalentus mūsų sprendiniui.

\newpage

\subsection{Integralinės kreivės}

\begin{figure}[h!]
    \centering
    \includegraphics[width=0.5\textwidth]{1-updated.png}
    \caption{Lygties $2+y^2+\sqrt{5-x^2}yy'=0$ integralinės kreivės, kai $C$ reikšmė $1, 2, 3, 4$}
    \label{fig:pvz1}
\end{figure}

Kodas, sugeneruojantis integralinių kreivių pavaizdavimą 
taip pat buvo parašytas su python programavimo kalba 
naudojantis numpy bei matplotlib paketais.

\begin{lstlisting}[language=Python]
import numpy as np
import matplotlib.pyplot as plt

def y(x, C):
    return np.sqrt(C * np.exp(-2 * np.arcsin(x / np.sqrt(5))) - 2)

cs = [ 1, 2, 3, 4 ]
colors = [ 'cyan', 'magenta', 'blue', 'red' ]

plt.xlim(-np.sqrt(5), np.sqrt(5) * np.sin(0.5 * np.log(2)))
plt.ylim(-8, 8)
plt.xlabel("x")
plt.ylabel("y")

for c, color in zip(cs, colors):
    x_start = -np.sqrt(5) + 0.01
    x_end = np.sqrt(5) * np.sin(0.5 * np.log(c / 2))
    xs = np.linspace(x_start, x_end, 1000)
    ys = y(xs, c)
    plt.plot(xs, ys, color=color, label=f'C={c}')
    plt.plot(xs, -ys, color=color)
plt.legend()
\end{lstlisting}

\newpage
\section{Antras pratimas}

Turime paprastąją pirmos eilės diferencialinę lygtį:

\begin{equation}
    3x^3y'=y(3x^2-y^2)
\end{equation}

\subsection{Sprendimas}

Galime pastebėti, kad tai yra pirmos eilės homogeninė lygtis:

\begin{equation}
\begin{split}
\frac{dy}{dx}=\frac{y}{x}-\frac{1}{3}\left(\frac{y}{x}\right)^3
\end{split}
\end{equation}

Įvedame keitinį $u=\frac{y}{x}$ arba $y=ux$, tada $y'=u'x+u$ ir lygtis tampa:

\begin{equation}
\begin{split}
u'x+u=u-\frac{1}{3}u^3\\
u'x=-\frac{1}{3}u^3\\
\int\frac{du}{u^3}=-\frac{1}{3}\int\frac{dx}{x}, u\ne 0\rightarrow y\ne 0\\
\frac{u^{-2}}{-2}=-\frac{1}{3}\ln|x|-\frac{1}{3}\ln|C|\\
u^{-2}=\frac{2}{3}\ln|Cx|
\end{split}
\end{equation}

Gražinam keitinį:

\begin{equation}
\begin{split}
\frac{x^2}{y^2}=\frac{2}{3}\ln|Cx|\\
y^2=\frac{3x^2}{2\ln|Cx|}, C \neq 0
\end{split}
\end{equation}

Dar galime pastebėti, kad išmestas sprendinys $y=0$ tenkina pradinę lygtį.

\newpage
\subsection{Tikrinimas}

Prieš tikrindami randame išvestinę "patogesnėje" išraiškoje:

\begin{equation}
\begin{split}
(y^2)'=\left(\frac{3x^2}{2\ln|Cx|}\right)'\\
2yy'=\frac{3}{2}\left(\frac{2x\ln|x|-x}{\ln^2|Cx|}\right)
\end{split}
\end{equation}

Tikrinimą atliksime taip pat su "patogesne" lygties forma, padauginę abi puses iš $y$, tada:
\begin{equation}
\begin{split}
3x^3yy'=y^2(3x^2-y^2)\\
3x^3\frac{3}{4}\left(\frac{2x\ln|x|-x}{\ln^2|Cx|}\right)=\frac{3x^2}{2\ln|Cx|}\left(3x^2-\frac{3x^2}{2\ln|Cx|}\right)\\
\frac{18x^4\ln|x|-9x^4}{4\ln^2|Cx|}=\frac{9x^2}{2\ln|Cx|}-\frac{9x^2}{4\ln^2|Cx|}\\
\frac{18x^4\ln|x|-9x^4}{4\ln^2|Cx|}\equiv\frac{18x^4\ln|x|-9x^4}{4\ln^2|Cx|}
\end{split}
\end{equation}

\subsection{Palyginimas}

Sprendiniai palyginimui buvo rasti naudojant 
python programavimo kalbą su sympy paketu. 
Toliau pavaizduotas kodas naudotas rasti sprendinį:

\begin{lstlisting}[language=Python]
from sympy import *
x = symbols('x')
y = Function('y')(x)
equation = Eq(3 * x**3 * y.diff(x), y * (3 * x**2 - y**2))
solutions = dsolve(equation)
display(solutions[0], solutions[1])
\end{lstlisting}

Rastas sprendinys turi išraiška:

\begin{equation}
y=\pm\sqrt 3x\sqrt{\frac{1}{\widetilde{C}+2\ln(x)}}
\end{equation}

Nors iš pirmo žvilgsnio sprendiniai atrodo skirtingi, nesunku parodyti, 
kad išraiškos apibudina tas pačias kreives. Svarbu pastebėti, kad $C$ ir $\widetilde{C}$ reikšmės
skirsis, tačiau nėra sunku surasti transformaciją iš vienos konstantos į kitą.

Pirmiausiai pertvarkome automatiškai gautą sprendinį į panašesnę formą:

\begin{equation}
y^2=\left(\pm\sqrt 3x\sqrt{\frac{1}{\widetilde{C}+2\ln(x)}}\right)^2=\frac{3x^2}{\widetilde{C}+2\ln x}
\end{equation}

Galime pertvarkyti mūsų sprendinį į tokią pat formą:

\begin{equation}
y^2=\frac{3x^2}{2\ln Cx}=\frac{3x^2}{2\ln C + 2\ln x}
\end{equation}

Iš čia matom, kad $\widetilde{C}=2\ln C$

\newpage
\subsection{Integralinės kreivės}

Lygties $3x^3y'=y(3x^2-y^2)$ integralinės kreivės

\begin{figure}[h!]
    \centering
    \begin{minipage}{0.75\textwidth}
        \centering
        \includegraphics[width=.7\linewidth]{2-updated-my.png}
        \caption{Ranka gautas sprendinys}\label{fig:pvz2}
    \end{minipage}\hfill

    \begin{minipage}{0.75\textwidth}
        \centering
        \includegraphics[width=.7\linewidth]{2-updated-sympy.png}
        \caption{Sprendinys gautas su sympy}\label{fig:pvz3}
    \end{minipage}
\end{figure}

\newpage
Kodas naudotas sugeneruoti pirmą paveikslėlį:

\begin{lstlisting}[language=Python]
import numpy as np
import matplotlib.pyplot as plt

def y(x, C):
    return np.sqrt(3/2 * x**2 / np.log(C * x))

cs = [ 0.5, 1, 1.5, 2 ]
colors = [ 'cyan', 'magenta', 'blue', 'red' ]

plt.xlim(0, 4)
plt.ylim(-10, 10)
plt.xlabel("x")
plt.ylabel("y")

for c, color in zip(cs, colors):
    xs = np.linspace(1/c, 4, 10000)
    ys = y(xs, c)
    plt.plot(xs, ys, color=color, label=f'C={c}')
    plt.plot(xs, -ys, color=color)
plt.legend()
\end{lstlisting}

\newpage
Kodas naudotas sugeneruoti antrą paveikslėlį:

\begin{lstlisting}[language=Python]
import numpy as np
import matplotlib.pyplot as plt
from sympy import * 

x = symbols('x')
y = Function('y')(x)
equation = Eq(3 * x**3 * y.diff(x), y * (3 * x**2 - y**2))
solutions = dsolve(equation)

cs = [ 0.5, 1, 1.5, 2 ]
colors = [ 'cyan', 'magenta', 'blue', 'red' ]

plt.xlim(0, 4)
plt.ylim(-10, 10)
plt.xlabel("x")
plt.ylabel("y")

for c, color in zip(cs, colors):
    adjusted_c = 2 * np.log(c)
    xs = np.linspace(1/c+0.001, 4, 1000)
    sol = solutions[0].rhs.subs('C1', adjusted_c)
    ys = np.array([ sol.subs(x, xi).evalf() for xi in xs ])
    plt.plot(xs, ys, color=color, label=f'$\widetilde{{C}}=2\ln {c}$')
    plt.plot(xs, -ys, color=color)
plt.legend()
\end{lstlisting}

\newpage
\section{Trečias pratimas}

\subsection{Sprendimas}

\begin{equation}
\begin{cases}
     y'+2y=x  \\
     y(0)=1 
\end{cases}
\end{equation}

Pirma sprendžiame homogeninį atvejį:

\begin{equation}
y'+2y=0
\end{equation}

Lygtis yra tiesinė, homogeninė ir su pastoviais koeficientais, todėl galime formuoti charakteringąją lygtį:

\begin{equation}
\begin{split}
\lambda+2=0 \Rightarrow \lambda=-2
\end{split}
\end{equation}

Tada bendrasis homogeninės lygties sprendinys turės formą:

\begin{equation}
y=Ce^{-2x}
\end{equation}

Nehomogeninį variantą sprendžiame konstantų variavimo metodu, darydami prielaidą, kad nehomogeninės lygties sprendinys turės formą:

\begin{equation}
    y = C(x)e^{-2x}
\end{equation}

Tada

\begin{equation}
\begin{split}
(Ce^{-2x})'+2Ce^{-2x}=x\\
C'e^{-2x}-2Ce^{-2x}+2Ce^{-2x}=x\\
C'e^{-2x}=x\\
C'=xe^{2x}\\
C=\int xe^{2x}dx\\
C=\frac{1}{2}\int xd(e^{2x})\\
C=\frac{1}{2}\left(xe^{2x}-\int e^{2x}dx\right)\\
C=\frac{1}{2}\left(xe^{2x}-\frac{1}{2}e^{2x}\right)+\widetilde{C}\\
C(x)=e^{2x}\left(\frac{x}{2}-\frac{1}{4}\right)+\widetilde{C}
\end{split}
\end{equation}

\newpage

Gauname bendrąjį ir atskirąjį sprendinius:

\begin{equation}
\begin{split}
y=C(x)e^{-2x}=\frac{x}{2}-\frac{1}{4}+\widetilde{C}e^{-2x},\widetilde{C}\in\mathbb{R}
\end{split}
\end{equation}

\subsection{Tikrinimas}

Išvestinė:

\begin{equation}
\begin{split}
y'=\frac{1}{2}-2\widetilde{C}e^{-2x}
\end{split}
\end{equation}

Tikriname:

\begin{equation}
\begin{split}
\left(\frac{x}{2}-\frac{1}{4}+\widetilde{C}e^{-2x}\right)'+2\left(\frac{x}{2}-\frac{1}{4}+\widetilde{C}e^{-2x}\right)=x\\
\frac{1}{2}-2\widetilde{C}e^{-2x}+x-\frac{1}{2}+2\widetilde{C}e^{-2x}=x\\
x\equiv x
\end{split}
\end{equation}

Koši sprendinys:

\begin{equation}
\begin{split}
1=y(0)\\
1=\frac{0}{2}-\frac{1}{4}+\widetilde{C}e^{-2*0}\\
1=\widetilde{C}-\frac{1}{4}\\
\widetilde{C}=\frac{5}{4}\\
y(x)=\frac{x}{2}-\frac{1}{4}+\frac{5}{4}e^{-2x}
\end{split}
\end{equation}

\newpage
\subsection{Palyginimas}

Raudona linija vaizduoja analitinį sprendinį. Melyni taškai vaizduoja sprendinį gautą naudojant skaitinius metodus. Žalias taškas žymi pradinę sąlygą $y(0)=1$.

\begin{figure}[h!]
    \centering
    \includegraphics[width=0.5\textwidth]{3.png}
    \caption{Koši sprendinys}
    \label{fig:pvz3}
\end{figure}

\end{document}