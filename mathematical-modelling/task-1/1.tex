\documentclass[11pt]{article}

\usepackage{graphicx}
\usepackage{amsmath}
\usepackage{mathtools}
\usepackage[T1]{fontenc}
\usepackage[lithuanian]{babel}

% Margins
\topmargin=-0.45in
\evensidemargin=0in
\oddsidemargin=0in
\textwidth=6.5in
\textheight=9.0in
\headsep=0.25in

\title{ Pirmas laboratorinis darbas}
\author{ Arnas Vaicekauskas }
\date{\today}

\begin{document}
\maketitle

\section{Pirmas pratimas}

\setlength{\jot}{10px}

$$
2+y^2+\sqrt{5-x^2}yy'=0
$$

Kintamieji atsiskiria atlikus porą elementarių pertvarkymų:

\begin{equation}
\begin{split}
\sqrt{5-x^2}y\frac{dy}{dx}&=-(y^2+2) \\
\frac{ydy}{-(y^2+2)}&=\frac{dx}{\sqrt{5-x^2}}
\end{split}
\end{equation}

Integruojame:
\begin{equation}
    \begin{split}
        \int\frac{ydy}{-(y^2+2)}&=\int\frac{dx}{\sqrt{5-x^2}} \\
        -\frac{1}{2}\int\frac{d(y^2+2)}{y^2+2}&=\arcsin{\frac{x}{\sqrt{5}}}+C\\
        -\frac{1}{2}\ln|y^2+2|&=\arcsin{\frac{x}{\sqrt{5}}}+C\\
        \ln|y^2+2|&=C-2\arcsin{\frac{x}{\sqrt{5}}}\\
        y^2&=Ce^{-2\arcsin{\frac{x}{\sqrt{5}}}} - 2
    \end{split}
\end{equation}

Prieš tikrindami randame išvestinę "patogesnėje" išraiškoje:

\begin{equation}
    \begin{split}
        (y^2)'&=(Ce^{-2\arcsin{\frac{x}{\sqrt{5}}}} - 2)' \\
        2yy'&=\frac{-2C}{\sqrt{5-x^2}}e^{-2\arcsin{\frac{x}{\sqrt{5}}}}\\
        yy'&=\frac{-C}{\sqrt{5-x^2}}e^{-2\arcsin{\frac{x}{\sqrt{5}}}}
    \end{split}
\end{equation}
\newpage
Tikriname:

\begin{equation}
    \begin{split}
        2+y^2+\sqrt{5-x^2}yy'=0\\
        2+Ce^{-2\arcsin{\frac{x}{\sqrt{5}}}} - 2+\sqrt{5-x^2}\frac{-C}{\sqrt{5-x^2}}e^{-2\arcsin{\frac{x}{\sqrt{5}}}}=0\\
        2-2+Ce^{-2\arcsin{\frac{x}{\sqrt{5}}}}-Ce^{-2\arcsin{\frac{x}{\sqrt{5}}}}=0\\
        0\equiv0
    \end{split}
\end{equation}

\end{document}