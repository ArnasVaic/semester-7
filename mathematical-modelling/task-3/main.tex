\documentclass[11pt]{article}

\usepackage{graphicx}
\usepackage{amsmath}
\usepackage[T1]{fontenc}
\usepackage[lithuanian]{babel}

\usepackage[
    a4paper,
    bindingoffset=0.2in,
    left=1in,
    right=1in,
    top=1in,
    bottom=1in,
    footskip=.25in]{geometry}

\title{ Trečias laboratorinis darbas}
\author{ Arnas Vaicekauskas }
\date{\today}

\begin{document}

\maketitle

\section{Analitinis sprendimas}

Turime antros eilės tiesinę nehomogeninę lygtį.

\begin{align}
x''+9x=4\sin(t)-\cos(4t) \label{eqs:nohom}
\end{align}

Pirmiausia spręsime homogeninį atvejį norint gauti bendrąjį sprendinį:

\begin{align}
    x''+9x=0 \label{eqs:hom}
\end{align}

Lygtis \eqref{eqs:hom} turi pastovius koeficientus, 
todėl galime formuoti charakteringąjį polinomą
\begin{align*}
    \lambda^2+9=0\implies\lambda=\pm3i
\end{align*}
Šiuo atveju bendrasis sprendinys $x_B(t)$ turės formą 
\begin{align}
x_B(t)=C_1\sin(3t)+C_2\cos(3t)
\end{align}
Nehomogeninę lygtį \eqref{eqs:nohom} spręsime
konstantų variavimo metodu, todėl:

\begin{align}
x(t)=C_1(t)\sin(3t)+C_2(t)\cos(3t) \label{eqs:varsol}
\end{align}

Pagal variavimo metodą, darome prielaidą, kad 
\begin{align}
C_1'\sin(3t)+C_2'\cos(3t)=0  \label{eqs:assumtion}
\end{align}

Prieš įstatant varijuotą bendrąjį sprendinį \eqref{eqs:varsol} apskaičiuosime $x''$ ir \\ $x''+9x$: 

\begin{align*}
x''&=(C_1'\sin(3t)+3C_1\cos(3t)+C_2'\cos(3t)-3C_2\sin(3t))'\\
&=(\underbrace{C_1'\sin(3t)+C_2'\cos(3t)}_0+3C_1\cos(3t)-3C_2\sin(3t))'\\
&=(3C_1\cos(3t)-3C_2\sin(3t))'\\
&=3C_1'\cos(3t)-9C_1\sin(3t)-3C_2'\sin(3t)-9C_2\cos(3t)
\end{align*}

tada

\begin{align*}
x''+9x&=3C_1'\cos(3t)-9C_1\sin(3t)-3C_2'\sin(3t)-9C_2\cos(3t)+9C_1\sin(3t)+9C_2\cos(3t)\\
&=3C_1'\cos(3t)-3C_2'\sin(3t)
\end{align*}

Taigi, dabar sprendžiame lygtį
\begin{align}
3C_1'\cos(3t)-3C_2'\sin(3t)=4\sin(t)-\cos(4t) \label{eqs:1}
\end{align}
Iš prielaidos \eqref{eqs:assumtion} galime išsireikšti $C_1'=-C_2'\frac{\cos(3t)}{\sin(3t)}$ ir įsistatyti į \eqref{eqs:1}

\begin{align}
-3C_2'\frac{\cos(3t)}{\sin(3t)}\cos(3t)-3C_2'\sin(3t)=4\sin(t)-\cos(4t) \notag\\
-3C_2'\left(\frac{\cos^2(3t)}{\sin(3t)}+\frac{\sin^2(3t)}{\sin(3t)}\right)=4\sin(t)-\cos(4t) \notag\\
-\frac{3}{\sin(3t)}C_2'=4\sin(t)-\cos(4t) \notag\\
C_2'=\frac{1}{3}\cos(4t)\sin(3t)-\frac{4}{3}\sin(t)\sin(3t) \label{eqs:c2d}\\
C_2=\frac{1}{3}\int\cos(4t)\sin(3t)dt-\frac{4}{3}\int\sin(t)\sin(3t)dt \label{eqs:c2}
\end{align}

Šiuos integralus išspręsime atskirai. Pradėsime nuo antro; naudodamiesi
tapatybe \\ $\sin(\alpha)\sin(\beta)=\frac{1}{2}(\cos(\alpha-\beta)-\cos(\alpha+\beta))$
galime suprastinti integralą:

\begin{align*}
\int\sin(3t)\sin(t)dt\\
\frac{1}{2}\int(\cos(2t)-\cos(4t))dt\\
\frac{1}{4}\int\cos(2t)d(2t)-\frac{1}{8}\int\cos(4t)d(4t)\\
\frac{1}{4}\sin(2t)-\frac{1}{8}\sin(4t)\\
\frac{1}{2}\sin(t)\cos(t)-\frac{1}{4}\sin(2t)\cos(2t)\\
\frac{1}{2}\sin(t)\cos(t)-\frac{1}{2}\sin(t)\cos(t)(\cos^2(t)-\sin^2(t))\\
\frac{1}{2}\sin(t)\cos(t)(1-\cos^2(t)+\sin^2(t))\\
\frac{1}{2}\sin(t)\cos(t)(\sin^2(t)+\cos^2(t)-\cos^2(t)+\sin^2(t))\\
\sin^3(t)\cos(t)
\end{align*}

Naudodami tapatybę $\sin(\alpha)\cos(\beta)=\frac{1}{2}(\sin(\alpha-\beta)+\sin(\alpha+\beta))$ galime išspręsti pirmąjį integralą:

\begin{align*}
\int\sin(3t)\cos(4t)dt\\
\frac{1}{2}\int(\sin(-t)+\sin(7t))dt\\
-\frac{1}{2}\int\sin(t)dt+\frac{1}{14}\int\sin(7t)d(7t)\\
-\frac{1}{2}\int\sin(t)dt+\frac{1}{14}\int\sin(7t)d(7t)\\
\frac{1}{2}\cos(t)-\frac{1}{14}\cos(7t)
\end{align*}

Galiausiai įsistatome gautas išraiškas atgal į \eqref{eqs:c2} ir gauname $C_2(t)$:

\begin{align*}
C_2&=\frac{1}{3}\left(\frac{1}{2}\cos(t)-\frac{1}{14}\cos(7t)\right)-\frac{4}{3}\sin^3(t)\cos(t)+\tilde{C_2}\\
C_2(t)&=\frac{1}{6}\cos(t)-\frac{1}{42}\cos(7t)-\frac{4}{3}\sin^3(t)\cos(t)+\tilde{C_2}
\end{align*}

Norint rasti $C_1(t)$ galim įstatyti \eqref{eqs:c2d} į jau minėtą formulę $C_1'=-C_2'\frac{\cos(3t)}{\sin(3t)}$:

\begin{align}
C_1'=-\frac{\cos(3t)}{\sin(3t)}\left(\frac{1}{3}\cos(4t)\sin(3t)-\frac{4}{3}\sin(t)\sin(3t)\right)\notag\\
C_1'=\frac{4}{3}\sin(t)\cos(3t)-\frac{1}{3}\cos(4t)\cos(3t) \label{eqs:c1d}\\
C_1=\frac{4}{3}\int\sin(t)\cos(3t)dt-\frac{1}{3}\int\cos(4t)\cos(3t)dt \label{eqs:c1}
\end{align}

Šiuos integralus taip pat išspręsime atskirai. Šį kart pradėsime nuo pirmojo. Čia dar kartą galime pritaikyti 
tapatybę $\sin(\alpha)\cos(\beta)=\frac{1}{2}(\sin(\alpha-\beta)+\sin(\alpha+\beta))$

\begin{align*}
\int\sin(t)\cos(3t)dt\\
\frac{1}{2}\int(\sin(-2t)+\sin(4t))dt\\
-\frac{1}{4}\int\sin(2t)d(2t)+\frac{1}{8}\int\sin(4t)d(4t)\\
\frac{1}{4}\cos(2t)-\frac{1}{8}\cos(4t)
\end{align*}

Antrajam integralui išspręsti galime panaudoti dar vieną tapatybę:

$$\cos(\alpha)\cos(\beta)=\frac{1}{2}(\cos(\alpha-\beta)+\cos(\alpha+\beta))$$

tada

\begin{align*}
\int\cos(4t)\cos(3t)dt=
\frac{1}{2}\int(\cos(t)+\cos(7t))dt=
\frac{1}{2}\sin(t)+\frac{1}{14}\sin(7t)
\end{align*}

Viską įstatę atgal į \eqref{eqs:c2} gauname, kad

\begin{align*}
C_1&=\frac{4}{3}\left(\frac{1}{4}\cos(2t)-\frac{1}{8}\cos(4t)\right)-\frac{1}{3}\left(\frac{1}{2}\sin(t)+\frac{1}{14}\sin(7t)\right)+\tilde{C_1}\\
C_1(t)&=\frac{1}{3}\cos(2t)-\frac{1}{6}\cos(4t)-\frac{1}{6}\sin(t)-\frac{1}{42}\sin(7t)+\tilde{C_1}
\end{align*}

Galiausiai nehomogeninės lygties \eqref{eqs:nohom} analitinis sprendinys yra

\begin{align}
x(t)=
\left(\frac{\cos(2t)}{3}-\frac{\cos(4t)}{6}-\frac{\sin(t)}{6}-\frac{\sin(7t)}{42}+\tilde{C_1}\right)\sin(3t)\\
+\left(\frac{\cos(t)}{6}-\frac{\cos(7t)}{42}-\frac{4\sin^3(t)\cos(t)}{3}+\tilde{C_2}\right)\cos(3t)
\end{align}

\end{document}

